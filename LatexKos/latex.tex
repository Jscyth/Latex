\documentclass[11pt]{article}
\usepackage{graphicx}
\usepackage[T1]{fontenc}
\title{Nihil}
\author{Jakub Kos}
\date{20.1.2020}
\begin{document}
\maketitle
\newpage
Praca ta będzie wykonana (mniej lub bardziej) w takiej kolejności:
\begin{enumerate}
\item rozdziały
\begin{itemize}
\item podrozdziały
\end{itemize}
\item tabele
\item wzory matematyczne
\item obrazek
\item spis spisów
\item odnośniki
\end{enumerate}
i nie będzie to nic sensownego.(nihil novi)\cite{lac}.
\newpage
\section{rozdziały}
\subsection{podrozdziały}
Jak widać rozdziały (oraz co ważniejsze podrozdziały) są widoczne.
\subsection{nawet więcej podrozdziałów}
A co lepsze, nawiet kilka podrozdziałów na raz działa.
\newpage
\section{tabele}
\begin{tabular}{||l|c|r||} 
\hline
\multicolumn{3}{||c||}{tabela}
\\ \hline
(1,1) & (1,2) & (1,3)\\
(2,1) & (2,2) & (2,3)\\ \hline
(3,1) & (3,2) & (3,3)\\
(4,1) & (4,2) & (4,3)\\ \hline
\end{tabular}

\begin{table}[h]
\caption{tabela 2}
\label{tab:tabela 2}
\begin{center}
\begin{tabular}{|c|l|l|l|l|}
\hline 1 & 2 & 3 & 4 & 5 \\ \hline \hline \hline
raz & dwa & trzy & cztery & pięć \\ \hline
one & two & three & four & five \\ \hline
\end{tabular}
\end{center}
\end{table}
\section{wzory matematyczne}
\begin{equation}
\frac{\sum_{n=1}^{100}n*x^{\pi }}{\lim_{n \to \infty}x^{n}}\leq x
\end{equation}
\begin{equation}
\frac{d}{dx}c^n=nx^{n-1}
\end{equation}
\cite{równania}
Ta funkcja jak widać również działa i nawet ładnie wygłąda.
\newpage
\section{obrazek}
\begin{figure}[!h]
\begin{center}
\includegraphics[keepaspectratio,width=40mm]{pic}
\caption{obrazek}
\end{center}
\end{figure}

\begin{figure}[!h]
\begin{center}
\includegraphics[angle=315,keepaspectratio,width=40mm]{pic}
\caption{obrazek skosem}
\end{center}
\end{figure}

\begin{figure}[!h]
\begin{center}
\includegraphics[angle=270,keepaspectratio,width=30mm]{pic}
\caption{obrazek bokiem}
\end{center}
\end{figure}
\newpage
\section{spisy}
\tableofcontents
\listoffigures
\listoftables
\newpage
\begin{thebibliography}{8}
\bibitem{lac}
(łacina) nic nowego

\bibitem{równania}
 https://www.codecogs.com/latex/eqneditor.php
\end{thebibliography}
\end{document}
